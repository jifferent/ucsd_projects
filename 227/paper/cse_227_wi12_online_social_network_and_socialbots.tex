\documentclass[12pt,a4paper]{article}

\usepackage[utf8]{inputenc}

\usepackage{fullpage}
\usepackage[top=2cm, bottom=2cm, left=4cm, right=4cm]{geometry}

\usepackage{url}

% This is now the recommended way for checking for PDFLaTeX:
\usepackage{ifpdf}

%\newif\ifpdf
%\ifx\pdfoutput\undefined
%\pdffalse % we are not running PDFLaTeX
%\else
%\pdfoutput=1 % we are running PDFLaTeX
%\pdftrue
%\fi

\ifpdf
\usepackage[pdftex]{graphicx}
\else
\usepackage{graphicx}
\fi


\title{Impact of fake profiles on social medias}
\author{
  \texttt{pierre-louis.gottfrois@epitech.eu} \\
  \texttt{pierre.fourgeaud@epitech.eu} \\
  \texttt{sacha.ott@epitech.eu}
}

\begin{document}

\ifpdf
\DeclareGraphicsExtensions{.pdf, .jpg, .tif}
\else
\DeclareGraphicsExtensions{.eps, .jpg}
\fi

\maketitle

\begin{abstract}
  Social media websites such as Facebook or Twitter have become increasingly popular as
  a way to meet and converse with friends, co-workers and even people you don't know.
  At the same time, security concerns inherent in such communication have been explored
  in various different ways. Also fake profiles is a serious security threat to human
  communication in social media.

  This paper presents the concern about social media bots and their goals in today's context.
  The paper further describes countermeasures in order to identify such threats.
\end{abstract}

\section{Introduction}
Online Social Networks (OSNs) such as Facebook and Twitter have successfully accomplished
their goal of connecting people together. With millions of active users using their
platforms every day, they have reach a point where third party companies want to exploit
them as an effective media to reach and potentially influence a large and diverse
population of web users. For example, there are now more than 850 million active Facebook
users, with more than 200 million added in 2011 and over 140 million active Twitter
users \cite{stats_of_the_day}.

It is now common for web users to share their personal and professional lives using OSNs.
Today’s users use Internet, cell phones and mobile devices every day to talk, socialize
and share “things” with their family, friends and colleagues. However, online social
experience is not exclusive to only human beings.

A new kind of computer programs called “socialbots” are now online, gathering huge amount
of informations and trying to socialize with real users in order to potentially influence them.
A socialbot is an automation software who will automatically take control of an online
account on a particular OSN. It has the ability to do basic actions such as posting
contents and sending connection requests to real users. Most advanced socialbots are made
in a fashion way such as they can interact with real users without being spotted.
This allows the socialbot to “infiltrate” user’s connections in order to reach an
influential position, that is, to spread misinformation and propaganda in order to bias
the public opinion, perform surveillance, and even more.

In this paper, we recovered data analytics from various sources that show today’s
statistics on OSNs. We presented different ways of how to recognize a socialbot on Facebook
in order to identify what are the recurrent patterns seen in fake profiles. From the user
side, we show that a lot of OSN users are not careful enough when dealing with connection
requests. Even more when they share mutual connections with the sender. This is one behavior
being exploited by socialbot makers to achieve a large-scale infiltration with a high success
rate. Finally, we presented first some basic techniques used by socialbot makers in
comparison of more advanced techniques used to have more accurate socialbots capable of almost
pass themselves as human beings.

In conclusion, we discussed the importance of the human factor in such threats and how OSN
designers are trying to deal with them.

\section{Preliminaries}

  \subsection{Online Social Networks}
  An Online Social Network (OSN) is an online service, platform or site that focuses on building
  and reflecting of social networks or social relations among people, who, for example, share
  interests and/or activities. A social network service consists of a representation of each
  user (often a profile), his/her social links, and a variety of additional services
  \cite{wikipedia_social_networking_service}.

  \subsection{Socialbots}
  A socialbot is an automation software who will automatically take control of an online account
  on a particular OSN. It has the ability to do basic actions such as posting contents and
  sending connection requests to real users. More advance socialbot are using heuristics and
  learned observations about user’s behavior in order to increase the magnitude of their potential
  damage. Socialbot’s main goal is to imitate real user’s behavior as close as possible and to
  perform scalable attacks in a particular OSN.

\section{Online Social Networks Statistics}

  \subsection{Facebook}
  Facebook is a social networking service and website launched in February 2004. After 8 years,
  Facebook has more than 845 million active users. A January 2009 study ranked Facebook as the
  most used social networking service by worldwide monthly active users
  \cite{wikipedia_facebook}.

  With 50.3\% of North America population, 57\% users on facebook said to talk to people more
  online than they do in real life. 48\% of young Americans said they find out about news
  through facebook \cite{facebook_takes_over_top_spot_twitter_climbs}.

  It is obvious that Facebook is a good place for socialbot makers. The huge number of users
  using facebook every day makes this OSN the place with the maximum likelihood to trap real
  users with fake accounts.

  \subsection{Twitter}
  Twitter is an online social networking service and micro-blogging service that enables its
  users to send and read text-based posts of up to 140 characters, known as "tweets".
  It was created in March 2006 and launched that July. The service rapidly gained worldwide
  popularity, with over 465 million users as of 2012, generating over 175 million tweet per day
  \cite{wikipedia_twitter}.

  It has been described as "the SMS of the Internet." Both private Internet users and public
  corporations have embraced the micro-blogging site to share news, photos, links and more.

  Once again socialbot makers are looking after Twitter's functionalities. As a matter of fact,
  interesting content will be re-tweet by users within 92\% compare to 84\% due to personal
  connections. We can see here that not only because the content might be interesting but also
  because users usually trust their social network and their friends. This is a crucial fact
  for socialbot makers to know as their goal is to act at large-scale.

\section{How to recognize a Socialbot ?}
In this section we extracted statistics from various sources showing what are the ways to
recognize fake profiles over real profiles on Facebook. As a matter of fact, we found that
97\% of fake profiles are female. Moreover, 58\% of them are interested in both men and
women against only 6\% for real people. Profile picture is also a good clue with almost
always a picture of an attractive women.

We also found that a majority of fake profiles do not update their social status. In fact 43\%
of them have never updated their Facebook status compared to 15\% of real people. Even more
interesting, on average, fake profiles' pictures have 136 "tags" every 4 pictures against an
average of 1 "tag" for real users. People interests also have big differences, the average
number of entertainment interests listed is 3 for fake profiles against 12 for real profiles.
Moreover, 68\% of fake profiles claim to have attended college over 40\% for real users.

Finally, the easiest way to detect fake profile is with the number of social connections
(friends). In deed, the average number is 130 friends for real people against 726 friends
for fake profiles. We can see here that there is a very large disparity between fake and
real profile on Facebook. However it may not be so obvious for people not aware how to
recognize fake profile from real ones.\cite{facebook_user_profiles}

\section{Socialbot makers techniques}

  \subsection{Objectives}
  There are two main objectives when creating a socialbot: (1) to carry out a large-scale
  infiltration campaign in the targeted OSN, and (2) to harvest private user's data. In order
  to achieve these goals, the socialbot need to connect to a large number of either random or
  targeted OSN users. Whereas collecting user's data create great opportunities for socialbot
  makers to do some phishing, spam or even collect monetary valuable data.

  \subsection{Construction}
  In this part, we will not cover the network architecture needed to handle large-scale
  socialbots. It is absolutely necessary that such network provide some way to minimize
  traffic to avoid detection.\\
  \\
  Building a socialbot involves first creating an attractive profile in the targeted OSN. After
  that the credentials are given to the socialbot in order to get full control over this profile.

  Second, research shows that the social attractiveness of a profile in an OSN is highly correlated
  to its neighborhood size. For example, 130 connections is the average on Facebook. Thus, in order
  to increase the social attractiveness of a socialbot, the adversary orders each socialbot to
  connect to at most $N_{avg}$ other socialbots.

  Third, it is well known that when two users share mutual connections, they are more likely to
  connect to each other. Therefor in order to improve the potential infiltration in the targeted
  OSN, the adversary orders each socialbot to connect to user profiles with whom it has mutual
  connections.

  Finally, whenever a socialbot is connected to a user profile, the adversary orders the socialbot
  to collects all accessible user's profile information in its neighborhood.

  In order to be effective, a socialbot has to meet two challenging goals: (1) socialbots need to
  be well developed in order to hide themselves from OSN, and (2) to implement heuristic that
  enable large-scale infiltration on targeted OSN.\cite{socialbot_network}

\section{Conclusion}
We have discussed what are the existing threats on OSN and saw various techniques on how to
recognize socialbots. We talked specially about Facebook because we believe it is the biggest
and more popular OSN right now with its millions of users. We saw what needed to be done in theory
to create a socialbot to take over at large scale on a targeted OSN. Unfortunately it is obvious
that socialbot makers are already taking advantages of these opportunities at large scale in order
to make money and/or to influence people in real world. We believe that such infiltration in OSNs
is only one over multiple future threats agains social networks and defending agains such threats
is the first step towards maintaining a safer social Web for all of us.

\section{Acknowledgments}
We would like to thank you all people for their feedback on an early draft of this paper.

\newpage

\bibliography{references}{}
\bibliographystyle{plain}

\end{document}