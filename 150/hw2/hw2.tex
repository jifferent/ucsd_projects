%
%  CSE 150. Assignment 2
%
%  Created by Pierre Louis Gottfrois on 2012-01-29.
%  Copyright (c) 2012. All rights reserved.
%
\documentclass[]{article}

% Use utf-8 encoding for foreign characters
\usepackage[utf8]{inputenc}
\usepackage{color}

% Setup for fullpage use
\usepackage{fullpage}

% More symbols
\usepackage{amsmath}
\usepackage{amssymb}
\usepackage{pstricks,pst-node,pst-tree}

% Surround parts of graphics with box
\usepackage{boxedminipage}

% Package for including code in the document
\usepackage{listings}

% If you want to generate a toc for each chapter (use with book)
\usepackage{minitoc}

% This is now the recommended way for checking for PDFLaTeX:
\usepackage{ifpdf}

\ifpdf
\usepackage[pdftex]{graphicx}
\else
\usepackage{graphicx}
\fi
\usepackage{tikz,tkz-tab}

\title{CSE 150. Assignment 2}
\author{Pierre-Louis Gottfrois}

\date{2012-01-29}

\begin{document}

\ifpdf
\DeclareGraphicsExtensions{.pdf, .jpg, .tif}
\else
\DeclareGraphicsExtensions{.eps, .jpg}
\fi

\maketitle

\section{Probabilistic reasoning}
%\prod_{\substack{i=0}}^{k} P(Y_{i}=1|X=0)
\begin{align*}
  r_{k} & = \frac{P(X =1|Y_{1} =1, Y_{2} =1, . . . , Y_{k} =1)}{P(X =0|Y_{1} =1, Y_{2} =1, . . . , Y_{k} =1)} \\
  &= \frac{\frac{P(X=1)*\prod_{\substack{i=2}}^{k} P(Y_{i}=1|X=1)}{P(Y_{1}=1,Y_{2}=1, . . . , Y_{k}=1)}}{\frac{P(X=0)*\prod_{\substack{i=1}}^{k} P(Y_{i}=1|X=0)}{P(Y_{1}=1,Y_{2}=1, . . . , Y_{k}=1)}} \\
  &= \frac{\prod_{\substack{i=2}}^{k} P(Y_{i}=1|X=1)}{\prod_{\substack{i=1}}^{k} P(Y_{i}=1|X=0)} \\
  &= \frac{\prod_{\substack{i=2}}^{k} \frac{2^{i-1}+(-1)^{i-1}}{2^i+(-1)^i}}{\prod_{\substack{i=1}}^{k} 1-(1/2)} \\
\end{align*}

Here are some values of $x$ computed with $f(x)$: \\

\begin{tabular}{|c|c|}
  \hline
  $x$ & $f(x)$ \\
  \hline
  $1$ & $1.000$ \\
  \hline
  $2$ & $0.8$ \\
  \hline
  $3$ & $1.142$ \\
  \hline
  $4$ & $0.940$ \\
  \hline
  $5$ & $1.030$ \\
  \hline
  ... & ... \\
  \hline
  $31$ & $1.000$ \\
  \hline
\end{tabular}
\\
\\
It depends on the day of the month due to the product for $i=2$ and $i=1$ on both side of the division, to $k$. \\
The diagnosis seems to become less certain while $k$ becomes bigger. As a matter of fact, the more symptoms you have, the more uncertain it is to know what is the disease.
Assuming all the symptoms match the two forms of the disease.

\newpage
\section{Noisy-OR}

\begin{enumerate}
  \item $P(Z =1|X =0, Y =1) > P(Z =1|X =0, Y =0)$ \\
  \\
  \begin{align*}
    P(Z =1|X =0, Y =1) & = 1-(1-Px)^0*(1-Py)^1 \\
    &= 1-(1)(1-Py) \\
    &= 1-(1-Py) \\
    &= 1-1+Py \\
    &= Py
  \end{align*}
  \begin{align*}
    P(Z =1|X =0, Y =0) & = 1-(1-Px)^0*(1-Py)^0 \\
    &= 1-(1)*(1) \\
    &= 0
  \end{align*}

  \item $P(Z =1|X =1, Y =0) < P(Z =1|X =0, Y =1)$ \\
  \\
  \begin{align*}
    P(Z =1|X =1, Y =0) & = 1-(1-Px)^1*(1-Py)^0 \\
    &= 1-(1-Px)(1) \\
    &= Px
  \end{align*}
  \begin{align*}
    P(Z =1|X =0, Y =1) & = Py \\
  \end{align*}

  \item $P(Z =1|X =1, Y =1) > P(Z =1|X =1, Y =0)$ \\
  \\
  \begin{align*}
    P(Z =1|X =1, Y =1) & = 1-(1-Px)^1*(1-Py)^1 \\
    &= 1-(1-Px)(1-Py) \\
    &= 1-(1-Py-Px+(Px*Py)) \\
    &= Py+Px-(Px*Py)
  \end{align*}
  \begin{align*}
    P(Z =1|X =1, Y =0) & = Px \\
  \end{align*}
\end{enumerate}

\newpage
\section{Conditional independence}

\begin{enumerate}
  \item True
  \item False
  \item True
  \item True
  \item True
  \item False
  \item True
  \item False
  \item False
  \item False
\end{enumerate}

\section{Subsets}

\begin{enumerate}
  \item $P(A) = P(A|E,C,F)$
  \item $P(A|B) = P(A|B,E,C,F)$
  \item $P(A|B,D) = P(A|B,D,E,C,F)$
  \item $P(B) = P(B|F,D)$
  \item $P(B|A,E) = P(B|A,E,F,D)$
  \item $P(B|A,C,E) = P(B|A,C,E,F,D)$
  \item $P(C) = P(C|A)$
  \item $P(C|E,F) = P(C|E,F,A)$
  \item $P(C|B,D,E,F) = P(C|B,D,E,F,A)$
  \item $P(E) = P(E|A,D,F)$
\end{enumerate}

\newpage
\section{Node ordering}

\begin{enumerate}
  \item $P(D,C,A,F,E,B) = P(D)P(C|D)P(A|C,D)P(F|A,C,D)P(E|F,A,C,D)P(B|E,F,A,C,D)$ \\
  \\
  $P(B|E,F,A,C,D) = P(B|E,A,C,D)$ \\
  $P(E|F,A,C,D) = P(E|F,C,D)$ \\
  $P(F|A,C,D) = P(F|D)$ \\
  $P(A|C,D) = P(A|D)$ \\
  $P(C|D) = P(C)$ \\

  $P(D,C,A,F,E,B) = P(D)P(C)P(A|D)P(F|D)P(E|F,C,D)P(B|E,A,C,D)$ \\

  \begin{center}
    $
    \newline
    \psmatrix[colsep=1.5cm,rowsep=1.5cm,mnode=circle]
    D&C&A&F&E&B
    \ncarc[arcangle=-30]{->}{1,1}{1,3}
    \ncarc[arcangle=-30]{->}{1,1}{1,4}
    \ncarc[arcangle=-30]{->}{1,1}{1,5}
    \ncarc[arcangle=-30]{->}{1,1}{1,6}
    \ncarc[arcangle=30]{->}{1,2}{1,6}
    \ncarc[arcangle=30]{->}{1,2}{1,5}
    \ncarc[arcangle=30]{->}{1,3}{1,6}
    \ncline{->}{1,4}{1,5}
    \ncline{->}{1,5}{1,6}
    \endpsmatrix
    $
    \newline
    \newline
    \newline
  \end{center}

  \item $P(C,A,F,B,D,E) = P(C)P(A|C)P(F|A,C)P(B|F,A,C)P(D|B,F,A,C)P(E|D,B,F,A,C)$ \\
  \\
  $P(E|D,B,F,A,C) = P(E|D,B,F,C)$ \\
  $P(D|B,F,A,C) = P(D|B,F,A)$ \\
  $P(B|F,A,C) = P(B|A,C)$ \\
  $P(F|A,C) = P(F)$ \\
  $P(A|C) = P(A)$ \\

  $P(C,A,F,B,D,E) = P(C)P(A)P(F)P(B|A,C)P(D|B,F,A)P(E|D,B,F,C)$ \\

  \begin{center}
    $
    \newline
    \psmatrix[colsep=1.5cm,rowsep=1.5cm,mnode=circle]
    C&A&F&B&D&E
    \ncarc[arcangle=-30]{->}{1,1}{1,4}
    \ncarc[arcangle=-30]{->}{1,1}{1,6}
    \ncarc[arcangle=30]{->}{1,2}{1,4}
    \ncarc[arcangle=30]{->}{1,2}{1,5}
    \ncarc[arcangle=-30]{->}{1,3}{1,5}
    \ncarc[arcangle=-30]{->}{1,3}{1,6}
    \ncline{->}{1,4}{1,5}
    \ncarc[arcangle=30]{->}{1,4}{1,6}
    \ncline{->}{1,5}{1,6}
    \endpsmatrix
    $
    \newline
    \newline
    \newline
  \end{center}

\end{enumerate}

\end{document}
